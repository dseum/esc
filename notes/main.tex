\documentclass[11pt, letterpaper]{article}

% Packages
\usepackage[margin=1in]{geometry}
\usepackage{setspace}
\usepackage[numbers]{natbib}
\usepackage{listings, listings-rust}
\usepackage{xcolor}
\usepackage{sectsty}
\usepackage[
  font=small,
  skip=6pt,
]{caption}
\usepackage[
  colorlinks = true,
  allcolors = blue,
]{hyperref}

\definecolor{codegreen}{rgb}{0,0.6,0}
\definecolor{codegray}{rgb}{0.5,0.5,0.5}
\definecolor{codepurple}{rgb}{0.58,0,0.82}
\definecolor{backcolour}{rgb}{0.95,0.95,0.92}

% Options
\allsectionsfont{\normalsize}
\lstset{
  backgroundcolor=\color{backcolour},
  commentstyle=\color{codegreen},
  keywordstyle=\color{magenta},
  numberstyle=\ttfamily\scriptsize\color{codegray},
  stringstyle=\color{codepurple},
  basicstyle=\ttfamily\footnotesize,
  breakatwhitespace=false,
  breaklines=true,
  captionpos=b,
  keepspaces=true,
  numbers=left,
  numbersep=7pt,
  showspaces=false,
  showstringspaces=false,
  showtabs=false,
  tabsize=2,
  framerule=0pt,
  lineskip=-0.1pt,
  frame=single
}

% Metadata
\title{Better SQL}
\author{Dennis Eum}
\date{January 12, 2025}

% Document
\begin{document}

\begin{lstlisting}[language=SQL]
INSERT INTO locations (name, description, latitude, longitude, geom)
VALUES (
    'Empire State Building',
    'Famous skyscraper in New York City',
    40.7484,
    -73.9857,
    ST_SetSRID(ST_MakePoint(-73.9857, 40.7484), 4326)
);
\end{lstlisting}

\begin{lstlisting}[language=Rust]
// PostGIS Extension Types
type Point;
type Polygon;
type Geometry = {Point, Polygon};
type ST_MakePoint(Float, Float): Point;
type ST_SetSRID(Geometry, Integer): Point;

stmt get_users(
  created_between: Option<Tuple<Integer, Integer>>,
  limit: Option<Integer>
): Array<Row> {
  `SELECT * FROM users`
  if created_between {
    `WHERE created_at BETWEEN {created_between[0]} AND {created_between[1]}`
  }
  if limit {
    `LIMIT {limit}`
  }
}

stmt bulk_add_users(usernames: Array<String>): Array<Row> {
  `INSERT INTO users (name) VALUES`
  for un in usernames; `,` {
    `({un})`
  }
  `RETURNING *`
}
\end{lstlisting}

\begin{lstlisting}[language=SQL]
SELECT [ ALL | DISTINCT [ ON ( expression [, ...] ) ] ]
  [ { * | expression [ [ AS ] output_name ] } [, ...] ]
  [ FROM from_item [, ...] ]
  [ WHERE condition ]
  [ GROUP BY [ ALL | DISTINCT ] grouping_element [, ...] ]
  [ HAVING condition ]
  [ WINDOW window_name AS ( window_definition ) [, ...] ]
  [ { UNION | INTERSECT | EXCEPT } [ ALL | DISTINCT ] select ]
  [ ORDER BY expression [ ASC | DESC | USING operator ] [ NULLS { FIRST | LAST } ] [, ...] ]
  [ LIMIT { count | ALL } ]
  [ OFFSET start [ ROW | ROWS ] ]
  [ FETCH { FIRST | NEXT } [ count ] { ROW | ROWS } { ONLY | WITH TIES } ]
  [ FOR { UPDATE | NO KEY UPDATE | SHARE | KEY SHARE } [ OF from_reference [, ...] ] [ NOWAIT | SKIP LOCKED ] [...] ]
\end{lstlisting}

\begin{lstlisting}[language=SQL]
INSERT INTO table_name [ AS alias ] [ ( column_name [, ...] ) ]
  [ OVERRIDING { SYSTEM | USER } VALUE ]
  { DEFAULT VALUES | VALUES ( { expression | DEFAULT } [, ...] ) [, ...] | query }
  [ ON CONFLICT [ conflict_target ] conflict_action ]
  [ RETURNING { * | output_expression [ [ AS ] output_name ] } [, ...] ]
\end{lstlisting}

\begin{lstlisting}[language=SQL]
// Should accept
SELECT *
FROM notes
WHERE created_at BETWEEN '2024-02-01 00:00:00' AND '2024-02-15 23:59:59'
  AND content = '';

// Should reject
SELECT *
FROM notes
WHERE created_at BETWEEN '2024-02-01 00:00:00'
  AND content = '';
\end{lstlisting}

A query has to be checked against:
\begin{itemize}
  \item All referenced tables and their columns have to exist. Use a symbol table per query to check aliases.
  \item All referenced columns must be of the correct type. Compare the type of the column with the type of the desired expression.
  \item Order of the clauses matters. Not checked if parsers accepts yet, but based on its specification, it should.
  \item \lstinline{WHERE} clause should allow most \href{https://www.postgresql.org/docs/9.0/functions.html}{operators}.
  \item Default values for insert
\end{itemize}

Questions:

\begin{itemize}
  \item Datatype of timestamp vs text, how to define?
  \item How should types be defined for extensions/parsed?
\end{itemize}

\bibliographystyle{IEEEtranN}
{\footnotesize\bibliography{index}}

\end{document}
